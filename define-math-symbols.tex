% Para fórmulas matemáticas
\usepackage{amsmath}
\usepackage{amssymb}
\usepackage{mathtools}

% Para setas usadas em fórmulas dos grafos
\usepackage{MnSymbol}

% Notações matemáticas usadas com frequência no texto.
% Isso possui três vantagens:
% 1 - Dá menos trabalho digitar as fórmulas
% 2 - Dá mais consistência à notação no trabalho
% 3 - Se mudar de ideia qto à notação, é só trocar aqui
%     e o trabalho todo fica certo.
% Notação matemática
\newcommand{\R}{\mathbb{R}} % Reais
\renewcommand\vec{\mathbf} % vetor como negrito
\newcommand{\avg}[1]{\left\langle #1 \right\rangle} % ... média
\newcommand{\defn}{\coloneqq} % Usamos "=", ":=" ou "\equiv?"
\newcommand{\noloop}[1]{#1^\nlcirclearrowleft} % ... sem loops
% Redes direcionadas
\newcommand{\linkin}[1]{#1^\leftarrow} % ... entrada
\newcommand{\linkout}[1]{#1^\rightarrow} % ... saída
\newcommand{\linkboth}[1]{#1^\leftrightarrow} % ... direcionada
% Redes ponderadas e direcionadas
\newcommand{\win}{w^\leftarrow} % peso e direção de entrada
\newcommand{\wout}{w^\rightarrow} % peso e direção de saída
\newcommand{\weighted}[1]{#1^w} % ... com pesos
\newcommand{\weighteddir}[1]{#1^{w\rightarrow}} % ... com peso e direção
% Reciprocidade
\newcommand{\recin}[1]{#1^\leftlsquigarrow} % ... entrada não-recíproca
\newcommand{\recout}[1]{#1^\leadsto} % ... saída não-recíproca
\newcommand{\recboth}[1]{#1^\leftrightsquigarrow} % ... recíproca
% Capacidades
\newcommand{\capac}{\textit{cap}}